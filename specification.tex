\documentclass{article}
\usepackage[utf8]{inputenc}
\usepackage[T1]{fontenc}
\usepackage{geometry}



\title{Software Engineering -- specification}
\author{% alphabetical ordering: feel free to change it at will
	\textsc{Antoine GRIMOD},
	\textsc{Arnaud DABY-SEESARAM},
	\textsc{Gabriel JANTET} and
	\textsc{Zibo YANG}}
\date{September 2021}

\newif\ifrts
\def\rts{\ifrts RTS\else Real Time Strategy game (RTS)\rtstrue\fi}
\def\ie{\textit{i.e.}}

\begin{document}
\maketitle

\section{Project introduction}
The project is to build a online multiplayer \rts. In this project, we intend to write
two main applications: a server and a client.

The \rts~will take place in a fantasy world in which each player evolves.
Their main goal is to build and continuously improve a village or small town,
and raise and interact with other villages and towns.
The players controls and influence the number of inhabitants, their professions
and resources usage\footnote{An inhabitant will cost resources. Thus, the player
has to keep a balance between developing its place and increasing its size.}.
\medskip

The players will be able to interact with one another through:
\begin{description}
	\item[military campaigns] a player can raise an army and use it to attack
	another player
	\item[trades] two players can agree on a commercial trade
\end{description}



\section{Project management and technical choices}
\subsection{Project management}
% TODO reformulate ?
In order to keep track of the project advancement, we use a plain bare
repository (with no web-GUI) available at \texttt{git.sofamaniac.xyz:rts.git}.
The repository is accessibe for reading to anyone (everyone can clone it
using \texttt{git clone git://git.sofamaniac.xyz/rts}).

We will meet at least once a week during Software Engineering sessions and have
a Discord channel in order to stay in touch during the week.

\subsection{Programming languages and implementation tests}

We will probably\footnote{We might have a change of heart after the first two
weeks.} use:
\begin{itemize}
	\item Rust for the server-side
	\item Go for the client-side
\end{itemize}

None of us have ever used these languages in whole projects in the past, but we
are willing to discover them. This said, if the group decide that one of the
above language no longer suits the project aim or that another known language
would be a better solution, we would switch to the latest.
\medskip

There exist static analysers for both these languages. We intend to try some of
them on our code.
\medskip

Each distinct implementation task will be developed in separate git-branches.
These will be merged as soon as the features have been improved and stabilised.
An implementation will be considered over if it matches its specification and
passes the corresponding tests (which could be created for this implementation).

\subsection{Networking part}
\subsubsection{Protocol}
During the game, the server and the client might need to exchange data (map
parts, location of the players, \dots).

Due to the large number of clients, the server might have several requests at
the same time. The client might wait for several requests to complete as well,
especially during the launch, as the client will have to load the player's
location, their wealth, the map, \dots

\medskip

The server and the client will share the following protocol:
\begin{itemize}
	\item On the start: The client opens a TCP connection to the server.
	\item On the stop: The client sends an exit message and disconnects.
\end{itemize}

During the game, both the client and the server could use the following
protocol to interact:
\begin{enumerate}
	\item \(A\) sends ``Q\texttt{\{id\}}.\texttt{\{type\}}:\texttt{query
	string}'' to \(B\), where \texttt{\{id\}} should be replaces by a query id,
	which is a string containing no dot`.', and the \texttt{type} is the query
	type that is a string containing so semicolon ``:'' .

	For example: ''1634309700-ClientId`` is a valid placeholder for
	\texttt{\{id\}} and ``info'' is a valid placeholder for \texttt{\{type\}}.

	The available queries are given below.

	\item \(B\) sends ``A\texttt{\{id\}}.\dots'' to \(A\), where \texttt{\{id\}}
	refers to the same string as received in the ``Q'' string.

	\item \(A\) treats the answer from the server and send back
	``S\texttt{\{id\}}.\dots``, where \texttt{\{id\}} stands for the query id,
	and the \dots~replace the status (either ``ok'' or ``nok'').
\end{enumerate}
Where \(A\) represent the server (resp. the client), and \(B\) represent the
client (resp. the server)



\subsubsection{Tasks sharing between server and client}
Running the client should not require heavy computational resources or use a
large bandwidth. In order to respect that wish, this is the current repartition
of the tasks between the client and the server:
\begin{description}
	\item[client]
		Stores:
			\begin{itemize}
				\item The state of the parts map it has already visited
				\item The position of some constructions
			\end{itemize}

		Computes:
			\begin{itemize}
				\item The local graphical effects
				\item The movement of the player as long as the map to print
				remains within the second boundary\footnote{See the description
				of the map synchronisation between the client and the server.}.
			\end{itemize}
	\item[server]
		Stores:
			\begin{itemize}
				\item The state of the map at all time
				\item The position of the players
				\item The possessions of the players
			\end{itemize}

		Computes:
			\begin{itemize}
				\item The remaining time for buildings
				\item Effects of map-wide actions
			\end{itemize}
\end{description}

\medskip

This part of the specification is likely to quickly evolve. This is a list of
improvement and changes that may be implemented some day:
\begin{itemize}
	\item Players have a cached version of what is on the map (for the parts
	they have already walked), but the map might have changed between two visits
	of a player.

	To manage this evolution, these three solutions might work:
	\begin{itemize}
	\item Send the server some information it has on the map, to check its
	validity
	\item The server stores the previous positions of the players and queries
	them to delete their cache of the updated parts of the map.
	\item The client forgets the cached map of places that are outside the
	third boundary\footnote{See the description of the map synchronisation
	between the client and the server.}.
	\end{itemize}
\end{itemize}

\subsection{Application packaging}

The final work will be packaged:
\begin{itemize}
	\item in an archive for a manual installation
	\item for several GNU/Linux-based OS
	(at least ArchLinux, Debian and GuixOs will be included)
\end{itemize}



\section{Game description}
\subsection{Species}
Each player will choose a species for its character at the beginning of the game.
They will choose among:
\begin{description}
	\item[elves]
	Elves have access to magic. They will be able to develop new powers, nurture
	them and use them.
	These powers would help the player kick start the rebuilding of its village at first
	(if any damage was to be caused by attacks), and eventually help the player
	defend the village, using defensive spells.

	An elf player will start its constructions slowly, but would store up a
	capacity to start better in case of destruction, as magic would then kick
	start the player.

	\item[humans]
	The humans will have access to technology. This technology would bring them stability
	as they will be able to both defend themselves from enemies and computerise their blue
	collar jobs such as mining or manufacturing clothes.

	As for the elfic magic, the human technology would take time to unlock completely,
	as the use of a piece of machinery for a certain amount of time might unlock a
	new system.

	\item[orcs]
	The orcs will be more resilient than the above species, allowing them to build
	heavier structures and faster than the other players. This strength will allow a
	player to start the game faster.

	Nonetheless, the orcs will not have tools to automatise the defence or the manufactures,
	which means that such tasks will always monopolise inhabitants.
\end{description}


\subsection{Town specification}
This section aims to give specifics on the different building types and the way a player can
unlock new and better ones. It will be divided in, four sections, three ow which will cover the
species specific structures a player can build and use.

\subsubsection{Common structures}
\begin{description}
    \item[town hall] to build any other structure and give order to the population (\ie~controlling
    how many citizen works in such and such area).

    In case of an attack, if the town hall is destroyed, it needs to be rebuilt before any other
    building.

    \item[mines, forests, \dots] in order to gain resources, which are required to build the
    structures and possibly trade with other players.

    At the beginning of the game, each player will discover basic resources and inhabitants from
    and with which build a town hall and start expoiting those resources.

    \medskip
    The forest will slowly grow as long as there exists a living tree. Thus, the player needs to
    be careful on how they use the resources available.
\end{description}

\subsubsection{Elf village specific structures}
\begin{description}
    \item[Mana Tower] Automagically gather mana
    \item[Academy of Magic]  Allocating inhabitants there allows the player to unlock new spells / abilities
\end{description}
\subsubsection{Human village specific structures}
\begin{description}
    \item[Research centre] By allocating inhabitants to the research centre, the player can collect
    techno-points which can be used to unlock new technologies
    \item[Automation centre ?] same as research centre but used to unlock automation upgrades (could be a building unlocked by research)
\end{description}
\subsubsection{Orc village specific structures}
\begin{description}
    \item[Orc breeding centre] Orcs inhabitants and soldier should be fast to produce, maybe add a building that automatically create an orc every few seconds
\end{description}

\subsection{Military specification}
This section aims to give specifics on the different soldier types and the way a player can
unlock new and better ones. It will be divided in, four sections, three ow which will cover the
species specific military soldiers and structures a player can use.

\subsubsection{Soldiers}
We might specialize the units depending on their race, giving them bonuses and maluses
For example for the first basic unit
Men : att: +0; speed:+0; def:+0
Elves: att:+0; speed:+1; def:-1
Orcs: att:+1; speed: -1; def:+0

\section{Other}
The first tasks of the group is to define the elements of the game:
\begin{enumerate}
	\item The different building types, professions, \dots~for the every day
	virtual life,
	\item The specification of the army and its evolution,
	\item The protocol to use and the repartition of the information between the
	server and the client \ie~decide what will the server and the client each
	store.
\end{enumerate}

Each of these three points lead to two implementations: one on the server-side
and one on the client-side.
\smallskip

The group members are requested to write and complete the documentation related
to their code, which is a way to split the work and ensures the correction of
what is written.



\bigskip
Constraints:
\begin{itemize}
	\item Provide a specification of the project: explain the main goal of the project, present the tasks distribution within the team, the temporary schedule, \dots
	\item Produce a documentation
	\item Provide testing tools
	\item Perform static analysis on the code
	\item Package the final work
\end{itemize}
\begin{itemize}
    \item The controls should be fully customisable
    \item The client must be resource efficient
\end{itemize}

\section{Planning}
\begin{itemize}
	\item October 1: Git tutorial
	\item October 5: Give our teachers an access to the git repository (which should contain the description / specification of the project: \~5 pages)
	\item October 8: Presentation of the project to the other groups
\end{itemize}

\end{document}

